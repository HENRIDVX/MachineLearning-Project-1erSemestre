\documentclass[12pt,a4paper]{report}

% ---------- PACKAGES ----------
\usepackage[utf8]{inputenc}
\usepackage[T1]{fontenc}
\usepackage{lmodern}
\usepackage{graphicx}
\usepackage{amsmath, amssymb}
\usepackage{hyperref}
\usepackage{geometry}
\usepackage{booktabs}
\usepackage{float}
\geometry{margin=2.5cm}

\hypersetup{
    colorlinks=true,
    linkcolor=blue,
    citecolor=blue,
    urlcolor=blue
}

% ---------- DOCUMENT ----------
\begin{document}

% ---------- TITLE PAGE ----------
\begin{titlepage}
    \centering
    {\Large Machine Learning Project}\\[1cm]
    {\Huge \textbf{Predicting the VIX Index Using\\
    Market and Macro Features}}\\[1.5cm]

    {\large Author(s): TODO Names}\\[0.5cm]
    {\large Program / Major: TODO}\\[0.5cm]
    {\large Date: \today}\\[2cm]

    \vfill
\end{titlepage}

\tableofcontents
\newpage

% ---------- ABSTRACT ----------
\chapter*{Abstract}
\addcontentsline{toc}{chapter}{Abstract}

In this project, we study whether the VIX index, often called the ``fear index'' of the US equity market, can be predicted using market and macroeconomic variables. 
We build a daily dataset from January 2020 to December 2023 using Yahoo Finance and other public sources. 
Our explanatory variables include the S\&P 500 index level and volume, commodity and energy prices (metals, oil, uranium), US Treasury yields, and the volatility-of-volatility index (VVIX).
We perform exploratory data analysis, correlation analysis and feature engineering (rolling volatility, SMA, RSI, and Principal Component Analysis for highly correlated assets).
We then train several regression models (Linear Regression, Random Forest, Gradient Boosting) to predict the VIX closing level.
Our results show low predictive power, with $R^2$ close to zero or negative on the test set.
We discuss the reasons for these poor performances (noise, time-series nature of the data, lack of lags) and outline possible improvements such as adding lagged variables, changing the prediction target, and using time-series specific models.

\newpage

% ---------- 1. INTRODUCTION ----------
\chapter{Introduction}

The VIX index measures the implied volatility of S\&P 500 options and is widely interpreted as a ``fear index'' for financial markets.
High levels of VIX are typically associated with periods of stress, uncertainty or crisis.

The goal of this project is to investigate whether the VIX can be predicted using market and macroeconomic indicators.
In particular, we aim to answer the following question:

\begin{quote}
\textit{Can we predict daily VIX levels using information from the equity market, commodity and energy markets, interest rates and the VVIX index?}
\end{quote}

This question is relevant for risk management and asset allocation: 
being able to anticipate changes in market volatility could help investors adjust their exposure to risky assets more efficiently.

This report is organised as follows.
In Chapter~\ref{chap:data} we describe the dataset and data sources.
In Chapter~\ref{chap:features} we present our exploratory data analysis and feature engineering.
Chapter~\ref{chap:models} describes the modelling approach and algorithms used.
In Chapter~\ref{chap:results} we discuss the experimental results.
Finally, Chapter~\ref{chap:conclusion} concludes and outlines possible improvements.

% ---------- 2. DATASET ----------
\chapter{Dataset and Data Sources}
\label{chap:data}

\section{Data Sources}

We collect daily financial data between 2020-01-01 and 2023-12-01 using the \texttt{yfinance} Python package and public CSV files.

\begin{itemize}
    \item \textbf{Target:} VIX index (\texttt{\^{}VIX}) closing level.
    \item \textbf{Equity market:} S\&P 500 index (\texttt{\^{}GSPC}) closing price and trading volume.
    \item \textbf{Commodities and metals:} Copper (HG=F), Aluminium (ALI=F), Gold (GC=F), Silver (SI=F), Nickel ETF (JJN), Zinc ETF (ZINC.L).
    \item \textbf{Energy:} Crude Oil (CL=F), Natural Gas (NG=F), Uranium ETF (URA).
    \item \textbf{Interest rates:} 10-year US Treasury yield (\texttt{\^{}TNX}) and 3-month T-bill (\texttt{\^{}IRX}).
    \item \textbf{Volatility-of-volatility:} VVIX index (\texttt{\^{}VVIX}).
    \item \textbf{S\&P 500 constituents:} list of companies and their market capitalisation from the DataHub S\&P 500 constituents dataset.
\end{itemize}

All series are aligned on trading days and merged using the date as an index.
Rows with missing values are dropped after feature construction.

\section{Descriptive Statistics}

TODO: briefly comment on the number of rows/columns, sample period, basic statistics (mean, std, min, max) and include a table or figure if needed.

% Example figure
\begin{figure}[H]
    \centering
    \includegraphics[width=0.9\textwidth]{sp500_close.png}
    \caption{Evolution of the S\&P 500 closing price over the sample period.}
    \label{fig:sp500_close}
\end{figure}

% ---------- 3. EXPLORATION & FEATURES ----------
\chapter{Exploratory Analysis and Feature Engineering}
\label{chap:features}

\section{S\&P 500 Returns and Volatility}

We compute daily returns of the S\&P 500 as
\begin{equation}
    r_t = \frac{P_t - P_{t-1}}{P_{t-1}} \times 100,
\end{equation}
where $P_t$ is the closing price at day $t$.
We also compute a 30-day rolling historical volatility, annualised by $\sqrt{252}$.

TODO: add plots of the return distribution and of the rolling volatility compared to the VIX.

\section{Correlation Analysis}

We perform a correlation analysis between VIX and all candidate variables using two approaches:
\begin{itemize}
    \item correlations on \textbf{closing levels},
    \item correlations on \textbf{daily returns}.
\end{itemize}

Interactive heatmaps are implemented using \texttt{ipywidgets} to vary the time window.
We observe that:
\begin{itemize}
    \item closing levels of several assets (e.g.\ gold, silver, aluminium, uranium, oil, interest rates, VVIX) exhibit non-negligible correlations with VIX;
    \item daily returns show almost no stable correlation with VIX, which is expected given their noisy behaviour.
\end{itemize}

Based on this analysis, we keep a subset of variables that are either economically meaningful or display interesting correlations with VIX.

\section{Engineered Features}

For the S\&P 500 we construct:
\begin{itemize}
    \item a 14-day Simple Moving Average (SMA) of the trading volume;
    \item a 14-day Relative Strength Index (RSI) computed on the volume.
\end{itemize}

Because some assets are strongly correlated (for example gold and silver, or oil and uranium), we apply Principal Component Analysis (PCA) to the pairs:
\begin{itemize}
    \item metals: (Gold, Silver) $\rightarrow$ Metal\_PC1, Metal\_PC2,
    \item energy: (Oil, Uranium) $\rightarrow$ Energy\_PC1, Energy\_PC2.
\end{itemize}

The first principal component in each pair captures most of the common variation (more than 80\%), while the second component is related to the spread between the two assets.

% ---------- 4. PROBLEM FORMALIZATION & MODELS ----------
\chapter{Problem Formalization and Models}
\label{chap:models}

\section{Supervised Learning Setup}

We formulate the task as a regression problem.
Let $X_t$ denote the vector of explanatory variables at day $t$, including:
\begin{itemize}
    \item VVIX closing level,
    \item S\&P 500 closing price and volume,
    \item SMA and RSI of S\&P 500 volume,
    \item Aluminium closing price,
    \item 10-year US Treasury yield,
    \item Metal\_PC1, Metal\_PC2,
    \item Energy\_PC1, Energy\_PC2.
\end{itemize}
The target variable is the VIX closing level $Y_t$ at the same date.

We split the dataset chronologically into a training set (80\%) and a test set (20\%), without shuffling, to respect the time-series structure.
Features are standardised using a \texttt{StandardScaler} fitted on the training set only.

\section{Models}

We implement three regression models:
\begin{itemize}
    \item \textbf{Linear Regression}: a simple baseline model with no regularisation.
    \item \textbf{Random Forest Regressor}: an ensemble of decision trees that can capture nonlinear effects and interactions.
    \item \textbf{Gradient Boosting Regressor}: an additive ensemble model that sequentially fits trees to the residuals of previous trees.
\end{itemize}

At this stage we use default hyperparameters from \texttt{scikit-learn} and do not perform an extensive hyperparameter search, which is left for future work.

% ---------- 5. RESULTS ----------
\chapter{Results and Discussion}
\label{chap:results}

\section{Evaluation Metrics}

We evaluate models using the Mean Squared Error (MSE), Root Mean Squared Error (RMSE) and coefficient of determination ($R^2$) on the test set.

\begin{table}[H]
    \centering
    \begin{tabular}{lccc}
        \toprule
        Model & MSE & RMSE & $R^2$ \\
        \midrule
        Linear Regression       & TODO & TODO & TODO \\
        Random Forest Regressor & TODO & TODO & TODO \\
        Gradient Boosting       & TODO & TODO & TODO \\
        \bottomrule
    \end{tabular}
    \caption{Test performance of the different models.}
    \label{tab:results}
\end{table}

TODO: fill in the metrics from your code.

\section{Interpretation of the Results}

In our experiments, the Linear Regression model achieves a very low $R^2$, close to zero, indicating that it explains only a small fraction of the variance of VIX.
The Random Forest and Gradient Boosting models even obtain negative $R^2$ scores on the test set, which means that they perform worse than a naive baseline that always predicts the mean VIX value.

These poor results suggest that:
\begin{itemize}
    \item the relationship between our features and the daily VIX level is weak and highly noisy;
    \item predicting the level of VIX at a daily frequency is intrinsically difficult;
    \item using contemporaneous variables (same-day information) without lagged features limits the predictive power.
\end{itemize}

% ---------- 6. LIMITATIONS & FUTURE WORK ----------
\chapter{Limitations and Future Work}

Several improvements could be explored:

\begin{itemize}
    \item \textbf{Lagged features:} include lagged values of VIX, VVIX, S\&P 500 returns and other variables to better capture temporal dynamics.
    \item \textbf{Alternative targets:} instead of predicting the level of VIX, predict its daily change or future realised volatility.
    \item \textbf{Hyperparameter tuning:} use grid search or random search to optimise the hyperparameters of Random Forest and Gradient Boosting.
    \item \textbf{Time-series specific models:} try ARIMA/GARCH models for volatility, or recurrent neural networks (LSTM/GRU) for multivariate time series.
    \item \textbf{Regularisation and feature selection:} test Lasso or Ridge regression to reduce overfitting and identify the most important predictors.
\end{itemize}

These directions would help move from a standard baseline towards a more advanced solution, as required in the project guidelines.

% ---------- 7. CONCLUSION ----------
\chapter{Conclusion}
\label{chap:conclusion}

In this project we investigated whether the VIX index can be predicted from market and macroeconomic variables such as equity indices, commodities, energy prices, interest rates and the VVIX index.
We built a daily dataset from 2020 to 2023, performed exploratory analysis and feature engineering, and trained several regression models.

Our models achieved low predictive performance on the test set, highlighting the difficulty of forecasting daily VIX levels with this type of information and without explicitly modelling time dependence.
Nevertheless, the project allowed us to practise the full machine learning pipeline: data collection, cleaning, feature engineering, dimensionality reduction, model training and critical evaluation.
Future work could focus on lagged features, alternative targets and more advanced time-series models.

% ---------- REFERENCES ----------
\begin{thebibliography}{9}

\bibitem{vix_cboe}
CBOE,
\textit{VIX White Paper},
Available online: \url{https://www.cboe.com/}.

\bibitem{yfinance}
Ran Aroussi,
\textit{yfinance: Download market data from Yahoo! Finance's API},
GitHub repository.

\bibitem{datahub_sp500}
DataHub,
\textit{S\&P 500 companies with financial information},
Available online.

% Add any scientific papers you decide to use as reference.

\end{thebibliography}

\end{document}
